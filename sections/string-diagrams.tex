\documentclass[../main.tex]{subfiles}
\begin{document}
    Causal reasoning tasks such as computing itnerventions and counterfactuals requires both a causal structure (e.g. a DAG) and consistent joint and conditional distributions (e.g. Markovian distributions). The structure can be thought of as syntax while the probability distributions play the role of semantics. Fixing a causal structure (e.g. fixing a DAG) is analgous to fixing the syntax and this constrains the allowed probability distributions(e.g. via the Markov conditions). The framework of categorical probability \cite{fong2013causal}, \cite{fritz2023free}, \cite{bojanczyk_causal_2019} make the distinction between syntax and semantics explicit. This framework is slightly more general than classical causal models, but perhaps more importantly the separation of syntax and semantics allows us to reason about problems on the structure level, e.g. by framing interventions as surgeries on the causal structures without having to deal with the probability distibutions. 

\end{document}