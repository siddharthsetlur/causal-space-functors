\documentclass[../main.tex]{subfiles}
\begin{document}
    \begin{defn}[Causal Space]
    \label{defn:Causal-Space}
        A causal space is a product probability space ${(\Omega, \mathcal{H}, \mathbb{P}) = (\times_{t\in T}\Omega_t, \otimes_{t\in T}\mathcal{E}_t, \mathbb{P})}$ equipped with a collection of Markov kernels ${\mathbb{K} = \{K_S: S\in 2^T\}}$ called a \emph{causal mechanism}. The Markov kernels $K_S: (\Omega_S, \mathcal{H}_S)\to (\Omega, \mathcal{H})$ satisfy the following two axioms:
        \begin{enumerate}
            \item for all ${A\in \mathcal{H}}$ and ${\omega_s\in \Omega_s}$,\begin{align*}
                K_{\emptyset}(\omega_S, A) &= \mathbb{P}(A);
            \end{align*}
            \item for all ${\omega_S\in \Omega_S}$, ${A\in \mathcal{H}_S}$, and ${B\in \mathcal{H}}$, 
            \begin{align*}
                K_S(\omega_S, A\cap B) &= 1_A(\omega_S)K_S(\omega_S, B) = \delta_{\omega_S}(A)K_S(\omega_S, B).
            \end{align*}
            In particular, for ${A\in \mathcal{H}_S}$, ${K_S(\omega_s,A) = \delta_{\omega_S}(A)}$.
        \end{enumerate}
    \end{defn}
    \begin{rem}[Observational and interventional distributions]
    \label{rem:Observational and interventional distribution}
        In \autoref{defn:Causal-Space}, the probability measure $\mathbb{P}$ plays the role of the \emph{observational distribution} in the SCM framework. The causal mechanisms provide a means of computing interventional distributions. In particular, the kernel $K_S$ allows us to determine what happens when we intervene on a subset $S\in 2^T$ of variables of the system. 
        
        Viewed through this lens, axiom 1 simply says that intervening on the empty set (not intervening at all) has no effect (the distribution on any event $A$ after not intervening is just the observational distribution $\mathbb{P}(A)$). Axiom $2$ is a bit trickier to unravel, so we rephrase it below using the intuition developed in a simple example we consider next.
    \end{rem}
    \begin{ex}[Two variable binary causal space]
    \label{ex:Two-variable-binary-causal-space}
        Let $T=\{X,Y\}$ be a set of two variables and ${\Omega_X = \Omega_Y = \{0,1\}}$. Consider the product probability space ${(\Omega_X\times \Omega_Y, \mathcal{E}_X\otimes \mathcal{E}_Y, \mathbb{P})}$, where ${\mathcal{E}_X}$ and ${\mathcal{E}_Y}$ are the power sets of ${\Omega_X}$ and ${\Omega_Y}$, respectively and let ${\mathbb{P}}$ be the uniform distribution on $\Omega$. Axiom $2$ constrains $K_X$ and $K_Y$. Recall that a Markov kernel $K$ between discrete spaces ${(\Omega, \mathcal{H})}$ and ${(\Omega', \mathcal{H}')}$ can be written as a ${|\Omega'|\times |\Omega\|}$ matrix, ${K_{ij} = K(\omega'_i, \omega_j)}$. Thus, we can write 
        \begin{align*}
            K_X = \begin{pmatrix}
               K_X(0, (0,0)) & K_X(1,(0,0))\\
                K_X(0, (0,1)) & K_X(1,(0,1))\\
                K_X(0, (1,0)) & K_X(1,(1,0))\\
                K_X(0, (1,1)) & K_X(1,(1,1)).
            \end{pmatrix}
        \end{align*}
        Axiom $2$ forces $K_X$ to be a block matrix of the form
        \begin{align*}
            K_X = \begin{pNiceArray}{c|c}
                K_X(0, (0,0)) & \Block{2-1}<\Large>{\mathbf{0}} \\
                K_X(0, (0,1)) \\
                \hline
                \Block{2-1}<\Large>{\mathbf{0}} & K_X(1, (1,0)) \\
                & K_X(1, (1,1))
              \end{pNiceArray}
        \end{align*}
        Indeed, we have $A = \{(0,0), (0,1)\}\in \mathcal{H}_X$ and so (by axiom 2) 
        \begin{align*}
            K_X(\omega_x, A) = \delta_{\omega_X}(A) = \begin{cases}
                1 & \text{if } \omega_X = 0,\\
                0 & \text{if } \omega_X = 1.
            \end{cases}
        \end{align*}
        Similarly, for $B = \{(1,0), (1,1)\}\in \mathcal{H}_X$, we have
        \begin{align*}
            K_X(\omega_x, B) = \delta_{\omega_X}(B) = \begin{cases}
                1 & \text{if } \omega_X = 1,\\
                0 & \text{if } \omega_X = 0.
            \end{cases}
        \end{align*}
        This yields the block structure above since ${K_X(0,B)=0}$ and ${K_X(1,A)=0}$ while the other two entries are non-zero and must sum to $1$. Intuitively, ${K_S(0,A)}$ is the probability distribution on ${A}$ after intervening by setting $X=0$. If ${A}$ does not contain an event where ${X=0}$, then the intervention makes ${A}$ impossible, hence the probability is $0$. Conversely, if $A$ contains all the events where $X=0$, then after intervening to set $X=0$, the probability of $A$ is $1$.
    \end{ex}
    \begin{lem}[Rephrasing axiom 2]
    \label{lem:Rephrasing-axiom-2}
        Axiom 2 of \autoref{defn:Causal-Space}  can be rephrased as follows: for all ${\omega_S\in\Omega_S}$ and ${B\in \mathcal{H}}$:
        \begin{align*}
            K_S(\omega_S, B) &=\begin{cases}
                 0 & \text{if } \omega_S\notin \pi_S(B)\\
                 K_S(\omega_{S}, B\cap \{\Omega_{T\setminus S}\times \omega_S\}) & \text{if } \omega_S\in \pi_S(B).
            \end{cases}
        \end{align*}
        where ${\pi_S(B)}$ denotes the projection of ${B}$ onto ${\Omega_S}$ \sid{I think we only need case $2$. If $\omega_S\notin \pi_S(B)$, then $B\cap \{\Omega_{T\setminus S}\times \omega_S\}=\emptyset$ and so $K_S(\omega_{S}, B\cap \{\Omega_{T\setminus S}\times \omega_S\})=0$ anyway.}
    \end{lem}
    \begin{proof}
        We first show that axiom 2 implies the statement above. Let ${\omega_S\in\Omega_S}$ and ${B\in \mathcal{H}}$ be arbitrary. Suppose ${\omega_S\notin \pi_S(B)}$ and set ${A = \{\omega_S\}\times \Omega_{T\setminus S}\in \mathcal{H}_S}$. Applying axiom 2, we get 
        \begin{align*}
              K_S(\omega_S, A\cap B) &= \delta_{\omega_S}(A)K_S(\omega_S, B).
        \end{align*}
        Due to ${\omega_S\notin \pi_S(B)}$, we have ${A\cap B = \emptyset}$  and consequently ${K_S(\omega_S, A\cap B) = 0 }$. On the right hand side, we have ${\delta_{\omega_S}(A) = 1}$ since ${\omega_S\in A}$. Thus, we conclude that ${K_S(\omega_S, B) = 0}$.

        Suppose now that ${\omega_S\in \pi_S(B)}$ and set ${A = \{\omega_S\}\times \Omega_{T\setminus S}\in \mathcal{H}_S}$. Applying axiom 2 again, we get
        \begin{align*}
            K_S(\omega_S, A\cap B) &= \delta_{\omega_S}(A)K_S(\omega_S, B) = K_S(\omega_S, B).
        \end{align*}
        Conversely, suppose the statement above holds and let ${\omega_S\in \Omega_S}$, ${A\in \mathcal{H}_S}$, and ${B\in \mathcal{H}}$ be arbitrary. We have two cases to consider. First, consider the case where ${\omega_S\in A}$. We need to use the rephrased axiom to show 
        \begin{align*}
            K_S(\omega_S, A\cap B) &= K_S(\omega_S, B)
        \end{align*}
        (since ${\delta_{\omega_S}(A)=1}$). If ${A\cap B = \emptyset}$, then in particular ${\omega_S\notin \pi_S(B)}$ (since ${\omega_S\in A}$ by assumption). Thus, by the rephrased axiom, we have ${K_S(\omega_S, B) = 0 = }K_S(\omega_S, \emptyset)$. Assume ${A\cap B\neq \emptyset}$ and apply the rephrased axiom to this set. 
        \begin{align*}
            K_S(\omega_S, A\cap B)&= K_S(\omega_S, B\cap A \cap \{\omega_S\times \Omega_{T\setminus S}\})\\
            &\stackrel{\spadesuit}{=} K_S(\omega_S, B\cap \{\omega_S\times \Omega_{T\setminus S}\})\\
            &\stackrel{\diamondsuit}{=}K_S(\omega_S, B).
        \end{align*}
        where ${(\spadesuit)}$ follows from the fact that ${A\cap \{\omega_S\times \Omega_{T\setminus S}\} = \{\omega_S\times \Omega_{T\setminus S}\}}$ (since ${\omega_S\in A}$) and ${(\diamondsuit)}$ follows from applying the rephrased axiom.\sid{The final step is more elegant if we use the definition of the axiom where we don't have the two cases (see my comment above). If we do have the two cases we just need to make a comment in the $\diamondsuit$ step that if the intersection is empty we get $0$ but this is OK since the intersection being empty means that $\omega_S\notin\pi_S(B)$ and so we get $K_S(\omega, B)$ eitherway}.  
        On the other hand, if ${\omega_S\notin A}$, we need to show that
        \begin{align*}
            K_S(\omega_S, A\cap B) &= 0.
        \end{align*}
        Let ${\tilde{B} = A\cap B}$. We can assume without loss of generality that ${\tilde{B} = \emptyset}$, since there is nothing to show otherwise (by definition we have ${K_S(\omega_S, A\cap B)=0}$). Note that ${\omega_S\notin\pi(\tilde{B})}$, since ${\omega_S\notin A}$. So, applying the rephrased axiom to ${\tilde{B}}$, we get ${K_S(\omega_S, \tilde{B}) = 0}$ as desired. 
    \end{proof}
    The takeaway here is that the interventional distribution of an event $B\in \mathcal{H}$ upon intervening on ${\omega\in\Omega_S}$ is completely determined by the intersection with events whose $\pi_S$-projection contains ${\omega_S}$. In particular any outcome ${\nu\in \Omega}$ with ${\nu_s\neq \omega_S}$ has probability $0$ in the interventional distribution ${K_S(\omega_S, \Omega)}$.
    \begin{ex}[Two variable binary example (contd)]\label{ex:Two-variable-binary-example-(contd)}
    Consider the causal space from \autoref{ex:Two-variable-binary-causal-space} again. Let ${B = \{(0,1),(1,1)\}}$ (i.e. the event $Y=1$). Computing $K_X(0,B)$ using the rephrased axiom we get 
    \begin{align*}
    K_X(0, B) &= K_X(0, B\cap \{(0,0),0,1\})\\
              &= K_X(0, \{(0,1)\})
    \end{align*}
    Moreover, 
    \begin{align*}
    K_X(0, (1,1)) = 0
    \end{align*}
    since ${0\notin \{1\} = \pi_X(\{(1,1)\})}$. Intuitively, this is what one would expect: the interventional distribution fixes the variable that we intervened upon. This result also allows us to define Markov kernels between variables ${X}$ and ${Y}$, as one often does in other frameworks of causal inference including structural causal models and string diagrams.  
    \end{ex}
    \begin{pro}[Kernels between subsets of variables]
    \label{pro:Kernels-between-subsets-of-variables}
    Let ${\mathfrak{C} = (\prod_{t\in T}\Omega_t, \otimes_{t\in T}\mathcal{E}_t, \mathbb{P}, \mathbb{K})}$ be a causal space and ${U, V\in 2^T}$ be be disjoint subsets of vriables. Then,
    \begin{align*}
        K_{U\to V}: \Omega_U\times \mathcal{H}_V &\to [0,1]\\
                (\omega_U, A)&\mapsto K_U(\omega_U, \prod_{i\in T\setminus \{U\cup V\}}\Omega_i\times \{\omega_U\}\times A_V)
    \end{align*}
    where ${A_V}$ denotes the projection ${\pi_V(A)}$ is a Markov Kernel that satisfies 
    \begin{enumerate}
        \item ${K_{U\to V}(\omega_U, A) = K_U(\omega_X, A)\foralll A \in \mathcal{H}_V}$ (compatibility with the original causal mechanism);
        \item ${K_{U\to W}(\omega_X, A) = K_{V\to W}\circ K_{U\to V}(\omega_X, A)\foralll A\in \mathcal{H}_Z}$ (compositionality).
    \end{enumerate}
    \end{pro}
    \begin{proof}
        Proving property $1$ is a direct application of \autoref{lem:Rephrasing-axiom-2}. Indeed, observe that ${A\in \mathcal{H}_V}$ is of the form ${\prod_{i\in \{T\setminus V\}}\Omega_i\times A_V}$. Since ${U\cap V=\emptyset}$, ${\pi_U(A) = \Omega_U\ni \omega_U}$. So, by \autoref{lem:Rephrasing-axiom-2},
        \begin{align*}
            K_U(\omega_U, A) &= K_U(\omega_U, A\cap \{\Omega_{T\setminus U}\times \omega_U\})\\
            &= K_U(\omega_U, \{\prod_{i\in \{T\setminus V\}}\Omega_i\times A_V\}\cap \{\Omega_{T\setminus U}\times \omega_U\})\\
            &= K_U(\omega_U, \{\prod_{i\in \{T\setminus U\cup V\}}\Omega_i\times A_V\times \omega_U\})\\
            &= K_{U\to V}(\omega_U, A). 
        \end{align*}
        This also proves that ${K_{U\to V}}$ is a Markov kernel, since $K_U$ is a Markov kernel. For every fixed ${A_0\in \mathcal{H}_V}$, the map ${\omega_U\mapsto K_{U\to V}(\omega_U, A) = K_{U\to V}(\omega_U, A) }$ is measurable since ${K_U}$ is measurable. For every fixed ${\omega_U\in\Omega_U}$, the map ${A\mapsto K_{U\to V}(\omega_U, A) = K_U(\omega_U, A)}$ is a measure since ${K_U}$ is a measure and it is a probability measure since 
        \begin{align*}
            K_{U\to V}(\omega_U, \Omega_V) &= K_U(\omega_U, \prod_{i\in T\setminus \{U\cup V\}}\Omega_i\times \{\omega_U\}\times \Omega_V)\\
            &= K_U(\omega_U, \omega_{T\setminus U}\times \omega_U)\\
            &\stackrel{\spadesuit}{=} 1.
        \end{align*} 
        where ${(\spadesuit)}$ is due to the fact that ${K_U(\omega_U, B)}$ is $0$ outside of the set of events ${B}$ whose projection onto ${\Omega_U}$ contains ${\omega_U}$ (by \autoref{lem:Rephrasing-axiom-2}). 
    \end{proof}
    \begin{rem}[Kernels between non-disjoint sets of variable]
    \label{rem:Kernels between non-disjoint sets of variables}
        
    \end{rem}
    It's worth spelling out what this boils down to when the sets are single variables. 
    \begin{cor}[Kernels between variables]\label{cor:Kernels-between-variables}
    Let ${\mathfrak{C} = (\prod_{t\in T}\Omega_t, \otimes_{t\in T}\mathcal{E}_t, \mathbb{P}, \mathbb{K})}$ be a causal space and ${X,Y\in T}$ be distinct variables. Then, 
    \begin{align*}
    K_{X\to Y}: \Omega_X\times \mathcal{H}_Y &\to [0,1]\\
                (\omega_X, A)&\mapsto K_X(\omega_X, \prod_{i\in T\setminus \{X, Y\}}\Omega_i\times \{\omega_X\}\times A_Y)
    \end{align*}
    where ${A_Y}$ denotes the projection ${\pi_Y(A)}$ is a Markov Kernel that satisfies 
    \begin{enumerate}
        \item ${K_{X\to Y}(\omega_X, A) = K_X(\omega_X, A)\foralll A \in \mathcal{H}_Y}$ (compatibility with the original causal mechanism);
        \item ${K_{X\to Z}(\omega_X, A_Z) = K_{Y\to Z}\circ K_{X\to Y}(\omega_X, A)\foralll A\in \mathcal{H}_Z}$ (compositionality).
    \end{enumerate}
    \end{cor}
    \begin{ex}[Kerneles between variables in the two binary variable case]
    \label{ex:Kernels-between-variables-in-the-two-binary-variable-case}
        Consider the causal space from \autoref{ex:Two-variable-binary-causal-space}. We can use \autoref{cor:Kernels-between-variables} to compute
        \begin{align*}
            K_{X\to Y}(\omega_X, A) &=  K_X(\omega_X, \omega_X\times A_Y)\\
                                    &=  \begin{pmatrix}
                                        K_X(0, \{(0,0)\}) & K_X(1, \{(1,0)\})\\
                                        K_X(0, \{(0,1)\}) & K_X(1, \{(1,1)\})
                                    \end{pmatrix}             
        \end{align*}
        where ${A\in \mathcal{H_Y}}$ is arbitrary. 
    \end{ex}
    % \begin{proof}
    %     Proving property $1$ also shows that ${K_{X\to Y}}$ is a Markov kernel, since ${K_X}$ is a Markov kernel. Property $1$ is a direct application of \autoref{lem:Rephrasing-axiom-2}. Indeed, let ${\omega_X\in \Omega_X}$ and ${A\in \mathcal{H}_Y}$ be arbitrary. Since ${A\in \mathcal{H}_Y}$ it is of the form ${\prod_{i\in \{T\setminus Y\}}\times A_Y}$ and so ${\omega_x\in \pi_X(A) = \Omega_X}$. Moreover 
    %     \begin{align*}
    %         A \cap \{\Omega_{T\setminus \{X\}}\times \omega_X\}  &= \{\prod_{i\in \{T\setminus Y\}}\Omega_i\times A_Y\} \cap \{\prod_{i\in \{T\setminus X\}}\Omega_i \times \omega_X\}\\
    %         &= \prod_{i\in T\setminus \{X,Y\}}\Omega_i \times A_Y \times \{\omega_X\}.
    %     \end{align*}
    %     By \autoref{lem:Rephrasing-axiom-2}, we have 
    %     \begin{align*}
    %         K_X(\omega, A) &= K_X(\omega_X, A \cap \{\Omega_{T\setminus \{X\}}\times \omega_X\})\\
    %                         &= K_X(\omega_X, \prod_{i\in T\setminus \{X,Y\}}\Omega_i \times A_Y \times \{\omega_X\})\\
    %                         &= K_{X\to Y}(\omega_X, A).
    %     \end{align*}
    % \end{proof}
\end{document}